\documentclass[12pt]{article}
\usepackage{mathpartir}
\usepackage[fleqn]{amsmath}
\usepackage{amssymb}
\usepackage{amsthm}
\usepackage{multicol}
\usepackage[utf8x]{inputenc}
\usepackage[french]{babel}
\usepackage{graphicx}
\usepackage[parfill]{parskip}
\usepackage{listings}
%\usepackage[margin=1.5in]{geometry}

\begin{document}
\pagestyle{empty}

\section*{Problème étudié}
Le langage OCaml propose différentes fonctionnalités assistant le programmeur
dans l'écriture de programmes robustes. Parmi celles-ci, le filtrage de motif
permet de s'assurer qu'aucun n'est oublié dans le traitement de données et que
l'on ne fait pas référence à des composantes indéfinies (un parallèle avec le
langage C serait la possibilité de garantir l'absence de déréférencement de
pointeurs nuls dans les valeurs manipulées par le code).

Toutes ces vérifications sont effectuées durant la phase de typage, puis les
informations nécessaires à ces analyses sont effacées : le langage source est
traduit en un langage intermédiaire non-typé, \emph{Lambda}. Conserver les
informations de typage dans la suite de la chaîne de compilation aurait
cependant plusieurs avantages : optimisations dirigées par les types,
opportunités d'analyses statiques, vérification de la correction des passes de
transformations de code, \emph{etc}.

Toutefois les motifs sont traduits très tôt en arbres de décisions --
séquences de \emph{if} et de \emph{switch}. Nous ne savons pas comment
conserver ou retrouver les informations issues du système de type dans cette
représentation de bas-niveau. C'est un pré-requis pour pouvoir faire de la
compilation typée sans changer le langage intermédiaire et donc l'architecture
du compilateur.

\section*{Solution proposée}
À partir d'une variante minimaliste du langage \emph{Lambda} conçue autour de
ces constructions problématiques, nous avons développé un système de type dans
lequel il est possible d'encoder les propriétés garanties par le filtrage de
motif d'OCaml.

Notre travail introduit une forme simple de \emph{dépendance} dans le système
de type : les types peuvent faire référence à des identificateurs de termes.
Cela permet de faire un lien entre des tests effectués à l'exécution sur des
valeurs et les informations que l'on peut apprendre statiquement sur le type
de ces valeurs.

\end{document}
