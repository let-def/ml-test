\section{Motivations}

Le sujet initial de ce TRE était une analyse statique du flot des exceptions
pour OCaml 4.00. 

Les systèmes d'exceptions permettent au programmeur de manipuler le contrôle de
flot beaucoup plus librement qu'avec un traditionnel appel de fonction
retournant une valeur. Malheureusement, il devient alors difficile de connaître
les différentes traces d'exécutions possibles car celles-ci ne correspondent
plus strictement à la structure du code source -- on parle de contrôle de flot
non-local.

Si le système de type d'OCaml permet de dire que les programmes bien typés ne
peuvent causer d'erreurs d'exécutions, les exceptions viennent affaiblir cette
affirmation.  Une exception non rattrapée causera l'arrêt d'un programme comme
aurait pu le faire une erreur d'exécution.  Pour garantir des programmes sûrs,
il est donc d'usage de se priver des exceptions.  L'analyse statique que nous
souhaitions effectuer visait à vérifier de manière mécanique l'absence
d'exceptions ou bien l'exhaustivité de leur traitement dans un code source.

Cette piste reprenait les résultats de la thèse de François Pessaux
\cite{ExcAnalysis} qui avait aboutie à un prototype permettant ce type
d'analyse pour OCaml 3.00.  Les résultats étaient très encourageants au point
de présenter un intérêt pour des applications industrielles.

Malheureusement, ce prototype n'a pas été porté vers les versions suivantes
d'OCaml et nous sommes arrivé à la conclusion que ce travail était trop
conséquent pour s'inscrire dans le cadre d'un TRE.  De plus, si l'approche
abordée dans la thèse permet une analyse fine, l'implantation est difficile à
maintenir car celle-ci s'appuie sur la syntaxe abstraite d'OCaml et implique de
développer en parallèle un typeur spécialisé.

\subsection{\emph{Lambda}, un langage intermédiaire d'OCamlc}

Notre travail de recherche s'est alors orienté vers le langage intermédiaire du
compilateur OCaml : \emph{Lambda}, le premier après le \emph{front-end}.

Cibler \emph{Lambda} apporte de nombreux avantages par rapport à un travail sur
la syntaxe abstraite, notamment :
\begin{description}
  \item[Indépendance vis-à-vis du \emph{front-end}.] Si le langage de surface 
    OCaml est régulièrement étendu, \emph{Lambda} est très stable et n'a pas
    connu de changements majeurs depuis l'introduction du système objet il y a
    plus de 10 ans. C'est une cible pérenne.
  \item[Langage minimaliste.]
    Les nombreuses constructions de surface sont réduites à un petit ensemble.
		La définition de \emph{Lambda} tient en une dizaine de lignes ;
		la définition du langage de surface en occupe des centaines.
    L'analyse peut ainsi se concentrer directement sur le cœur du langage et
    garder une taille modeste. Le travail de maintenance est moindre.
\end{description}

Ces avantages s'accompagnent des inconvénients suivants :
\begin{description}
  \item[Distance avec le langage source.] \emph{Lambda} étant issu d'une
    traduction du programme source après vérification des types, certaines
    informations sont perdues ou difficile à faire correspondre avec le
    programme original. 
  \item[Absence d'informations de types.] \emph{Lambda} n'est pas typé. Dans la
    chaîne de traitement du compilateur, les types sont vérifiés pour le
    langage source par le typeur puis effacés. Tout le reste du traitement
    s'effectue sur une version non-typée du programme. 
\end{description}

%\todo{figure : pipeline ocaml}.

Ce choix restreint les traitements possibles dans la suite de la compilation.
Le compilateur natif tente par exemple de réinférer des types dans le cadre de
certaines optimisations, mais il ne s'agit que d'approximation conservatrice.
Dans notre cas analyser le flot est beaucoup plus compliqué sans type.
% TODO: documenter effet latents

Concevoir une version de \emph{Lambda} préservant les types semble donc être un
prérequis pour un outil d'analyse viable.

\paragraph{Remarques} Le choix de l'effacement des types est justifié par
l'omniprésence du polymorphisme paramétrique dans le langage OCaml et de la
représentation très régulière des valeurs à l'exécution.

Le polymorphisme paramétrique permet de garantir qu'un programme bien typé est
sémantiquement équivalent après effacement des types. %\todo{ref didier rémy ?}
La représentation des valeurs permet de travailler correctement sans
information de types -- cela concerne par exemple la convention d'appel des
fonctions ou le passage du glaneur de cellule.

\subsection{\emph{Lambda}, un lambda-calcul}

%Le langage \emph{Lambda} est un lambda-calcul proposant des fonctions de très
%bas-niveau.

Le lambda-calcul, introduit par Church désigne une famille de langage de
programmation fonctionnelle reposant autour d'un unique mécanisme substitution.
Cela rend sa définition simplissime, le noyau syntaxique tient en trois règles :
$$t ::= x | \lambda x . t | t \, t$$
L'occurence d'une variable ; l'abstraction d'un terme par rapport à une variable ;
l'application d'un terme à un autre.
Mais derrière cette concision se cache une très grande expressivité : c'est un
langage turing-complet et d'un suffisamment haut niveau pour permettre un
codage léger des constructions classiques des langages de programmation
\cite{Landin}.

Un lambda-calcul est ainsi un choix naturel pour un langage intermédiaire, son
expressivité sera suffisante pour les évolutions futures du langage source.
Le \emph{Lambda} d'OCaml ajoute principalement des primitives d'accès mémoires
et de branchements; des extensions de bas-niveau.

\subsection{Un typage spécifique}

Les intérêts d'un langage intermédiaire typé sont multiples. Outre faciliter
l'analyse et l'optimisation des programmes, cela offre également des garanties
de correction. On peut ainsi vérifier que les passes de compilation et les
transformations intermédiaires appliquées au programme conserve certains
invariants. Le programme reste bien typé une fois exprimé dans un langage de
plus bas-niveau ou, inversement, une traduction erronée pourra être détectée
mécaniquement.

Plusieurs approches ont été étudiées pour la compilation typée.
\cite{MorrisettWCG99} propose un schéma de compilation typé depuis un
lambda-calcul vers un assembleur spécifique.

Une implantation majeure du langage de programmation fonctionnelle
Haskell, GHC \cite{Marlow98thenew}, a fait des choix très différents de ceux
d'OCaml dans la conception de son langage intermédiaire.  Ce dernier n'offre
pas de primitives de bas-niveau : les accès mémoires sont masqués derrière une
forme simplifiée de \emph{filtrage de motifs}. Mais il est typé et son système
de type, System $F_c$ \cite{Sulzmann07systemf}, permet de vérifier toutes
constructions d'Haskell (comme les \emph{GADTs}).

Les systèmes de types d'OCaml et d'Haskell présentent de nombreuses
similitudes. L'emploi réussi de System $F_c$ dans GHC nous a conforté dans la
faisabilité d'un traitement similaire dans OCaml.

Dans le souci d'être le moins intrusif possible dans le compilateur
OCaml actuel, il était nécessaire d'étendre le langage \emph{Lambda} et non de
le remplacer.  L'extension que nous proposons introduit un système de type avec
une forme de dépendance entre types et termes pour prendre en charge les
constructions de bas-niveau d'accès mémoire et de branchements de \emph{Lambda}.

C'est ce problème que nous avons spécifiquement cherché à résoudre et qui n'a,
à notre connaissance, pas été étudié dans le contexte d'un langage
intermédiaire.

%Pour mener à terme cet objectif, les étapes suivantes se profilaient :
%\begin{itemize}
%  \item concevoir un système de type s'inspirant de System $F_c$ et composant
%    avec les contraintes imposées par \emph{Lambda},
%  \item établir un codage des constructions d'Ocaml (types algébriques, modules,
%    GADTs, variants polymorphes) dans ce système afin de s'assurer que toutes
%    sont capturées,
%  \item étendre \emph{Lambda} et proposer un schéma de compilation vers ce
%    nouveau langage intermédiaire.
%\end{itemize}
%
%Ceci constitue un travail de grande ampleur. Dans ce rapport nous adressons le
%point précis de typer les contraintes imposées par \emph{Lambda}; un
%branchement et des accès mémoires beaucoup plus libres que ce que ne permet le
%\emph{filtrage de motif} traditionnel.

